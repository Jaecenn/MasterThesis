% arara: pdflatex: { synctex: yes }
% arara: makeindex: { style: ctuthesis }
% arara: bibtex

% The class takes all the key=value arguments that \ctusetup does,
% and a couple more: draft and oneside
\documentclass[twoside]{ctuthesis}

\ctusetup{
	preprint = \ctuverlog,
%	mainlanguage = english,
	titlelanguage = english,
	mainlanguage = english,
	otherlanguages = {czech,english},
	title-czech = {Analýza periodických vad textilních vláken ve frekvenční oblasti},
	title-english = {Analysis of the textile fibers unevenness in frequency domain},
	subtitle-czech = {},
	subtitle-english = {},
	doctype = M,
	faculty = F3,
	department-czech = {Katedra měření},
	department-english = {Department of Measurements},
	author = {Ondřej Renza},
	supervisor = {Ing. Jakub Parák},
%	supervisor-address = {Ústav X, \\ Uliční 5, \\ Praha 99},
%	supervisor-specialist = {John Doe},
	fieldofstudy-english = {Sensors and Instrumentation},
	subfieldofstudy-english = {Cybernetics and Robotics},
	fieldofstudy-czech = {Senzory a přístrojová technika},
	subfieldofstudy-czech = {Kybernetika a Robotika},
	keywords-czech = {zpracování digitálního signálu, textilní vady, nerovnoměrnost},
	keywords-english = {digital signal processing, textile defects, uneveness},
	day = 15,
	month = 5,
	year = 2016,
	specification-file = {ctutest-zadani.pdf},
%	front-specification = true,
%	front-list-of-figures = false,
%	front-list-of-tables = false,
%	monochrome = true,
%	layout-short = true,
}

\ctuprocess

\addto\ctucaptionsczech{%
	\def\supervisorname{Vedoucí}%
	\def\subfieldofstudyname{Studijní program}%
}

\ctutemplateset{maketitle twocolumn default}{
	\begin{twocolumnfrontmatterpage}
		\ctutemplate{twocolumn.thanks}
		\ctutemplate{twocolumn.declaration}
		\ctutemplate{twocolumn.abstract.in.titlelanguage}
		\ctutemplate{twocolumn.abstract.in.secondlanguage}
		\ctutemplate{twocolumn.tableofcontents}
		\ctutemplate{twocolumn.listoffigures}
	\end{twocolumnfrontmatterpage}
}

% Theorem declarations, this is the reasonable default, anybody can do what they wish.
% If you prefer theorems in italics rather than slanted, use \theoremstyle{plainit}
\theoremstyle{plain}
\newtheorem{theorem}{Theorem}[chapter]
\newtheorem{corollary}[theorem]{Corollary}
\newtheorem{lemma}[theorem]{Lemma}
\newtheorem{proposition}[theorem]{Proposition}

\theoremstyle{definition}
\newtheorem{definition}[theorem]{Definition}
\newtheorem{example}[theorem]{Example}
\newtheorem{conjecture}[theorem]{Conjecture}

\theoremstyle{note}
\newtheorem*{remark*}{Remark}
\newtheorem{remark}[theorem]{Remark}

% Abstract in Czech
\begin{abstract-czech}
To be done.
\end{abstract-czech}

% Abstract in English
\begin{abstract-english}
 To be done.
\end{abstract-english}

% Acknowledgements / Podekovani
\begin{thanks}
Děkuji.
\end{thanks}

% Declaration / Prohlaseni
\begin{declaration}
Prohlašuji, že jsem předloženou práci vypracoval samostatně, a že jsem uvedl veškerou použitou literaturu.

V Praze, \ctufield{day}.~\monthinlanguage{title}~\ctufield{year}
\end{declaration}

% Only for testing purposes
\listfiles
\usepackage[pagewise]{lineno}
\usepackage{lipsum,blindtext}
\usepackage{mathrsfs} % provides \mathscr used in the ridiculous examples

\begin{document}

\maketitle

\chapter{Introduction}
%\chapter{Introduction}
Textile manufacturing has always been important industry field. Major part of this industry is formed by process called spinning, where twisting strands of fibers together form yarn. The modern spinners - textile machines, that execute process of spinning - have been significantly improved and now they reach high level of automation. This allows not only faster and cheaper production but also more focus on quality of the produced textile yarn. 

Quality of the yarn could devaluate the final product by creating defects, such as a rapid changes in color or thickness etc., in textile material. Even with modern technologies it is still impossible to produce yarn without any defects. We can't prevent yarn defects by carefully selecting and preprocessing fiber, because some defects can be created by spinning process itself. Out of many types of defects this thesis is focused on analysis of yarn uneveness (also called yarn irregularity). This is describing yarn with a diameter that is not even along it's length, but it is changing it's value periodically. We can measure this defect in the form of mass variation per unit length.

Designed system is not aiming to improve quality of spinned yarn, but to monitor quality (specifically uneveness) of produced yarn during the spinning process. Due to this monitoring it is possible to stop spinning process if defective yarn is detected. This allows the operator to resolve the issues that caused it, e.g. by replacing spinner fiber source with new one and reconnecting different yarn endings.

The project - described in this thesis - has been made in cooperation with company Rieter CZ s.r.o, who provided the device requirements, critical measurement data and other important information. The goal of the project was to research and develop an algorithm for analysis and detection of yarn uneveness by using spectrograms and to design an embedded system capable of measuring diameter of yarn while fulfilling the required time constraints and implement detection algorithm to it. System is required to be controlled by ARM M4 microcontroller. Thus the algorithm has to be implemented in a way that takes in consideration memory and computational limitation of such microcontrollers.

Very important specification was requirement to analyse quality of two textile fibre types: the  yarn and the sliver. Where sliver is the input textile fibre for the spinning process and the yarn is it's final product. The both of which has significant physical differences. Mainly they differ in size, because yarn diameter is usually in range of micrometers and sliver diameter is in range from millimeters to centimeters. The diameter of the sliver is measured on combing machine, which precedes the spinning process. This requires different measuring system and filtration processing, therefore two systems and algorithms were designed for quality analysis of each fibre type. Their core content is equal but there are some major differences, which are described later in this thesis. The most significant difference is that processing of signal representing sliver diameter requires much more advanced techniques of digital signal processing. This is necessary due to strong presence of periodical artefacts on sliver diameter caused by machine preprocessing. This type of diameter fluctuations have to be distinguished from the actual sliver uneveness, which is task for complicated digital filtration in frequency domain.

\chapter{Theoretical Introduction}
Before the process of software and hardware development could begin, detailed theoretical research had to be done. Topics of the research include textile manufacturing, combing process, spinning process and textile fiber defects to understand better device requirements. Another step of research was focused on digital signal processing techniques that could be used on the project, mainly spectrogram estimation and it's calculation using Fast Fourier Transform, together with possible filtration algorithms. Another research topic was aimed to cover embedded systems and specifically micro-controller usage and it's limitations
\section{Textile Engineering}
Goal of textile manufacturing is to make fabric from textile fibers, which can be then used for clothes. This process can be separated in several stages:

- Preparatory Processes - prepares the textile fiber for spinning process by blending, carding and combing,

- Spinning - fibres are spun into yarns,

- Knitting or Weaving - yarns becomes fabric,

- Finnishing - fabric is transformed into clothes etc.

In regards to the topic of this thesis only two - the spinning and combing processes - are described.
\subsection{Textile Fibers}
Fibers are the basis for all textiles. We distinguished the two main types: natural fibers and synthetic fibers. The natural fibers are:

- Cotton - from the cotton plant,

- Linen - from the flax plant,

- Wool - from sheep and

- Silk - from silkworms.

Examples of widely used synthetic fibers are:

- Viscose - from pine trees and petrochemicals,

- Acrylic, nylon and polyester - from oil and coal. 

The cotton is the most important natural fiber and analysis of the quality in this thesis is is aimed specifically on cotton manufacturing. 
The fibers can have two main forms during the manufacturing process - the sliver and the yarn. The sliver is created by carding the fiber. In this process textile fibers are separated and then joint together into a loose strand of 1 cm to 4 cm in diameter. In the end of textile manufacturing process an yarn is created. It is a textile fiber with significantly smaller diameter, in comparison to sliver, usually in range of micrometers \cite{cite:FoFF}. 
\subsection{Combing Process}
Combing is a preparation process during textile manufacturing. It is sub-part of a cleaning process which precedes the spinning process. Cotton contains a lot of impurities such as dirt, dust, foreign materials, neps and very short fibers. All of these should be eliminated by cleaning process. The combing process removes mainly short fibers and neps in sliver, which helps to produce stronger and cleaner yarn.

Combing is used in production of medium-fine or fine yarns, where quality of yarn is important. This quality improvement is at cost of loss of raw material and high expenses for buying and operating the combing machines. 

Comber (see Image 1) consists of three main parts: 

- The Feed

- The Nipper

- The Comb

The process of removing the impurities is formed by attaching the input fibre in form of sliver to the feed roller

An input of this process is textile sliver (described in chapter xx).
\subsection{Spinning Process}
The term “spinning” in this context refers to the process that executes conversion of a large quantity of individual unordered fibers of relatively short length into a linear, ordered product of very great length by using spinning machines. There are three main methods of executing process of spinning:

- Ring Spinning,

- Rotor Spinning and

- Air-jet Spinning.

All of these systems yields yarn with different structures and properties. Each system has its advantages and limitations in terms of technical feasibility and economic viability.
\section{Overview of Yarn Quality Sensors}
Importance of yarn quality on final product became clear in 1950s when first electronic yarn quality sensors were invented. Since then many principles were used in detection of yarn defects - optical, mechanical or even chemical \cite{cite:1} . 
\subsection{State of Art}

\section{Digital Signal Processing}
Largest part of work on this project is oriented on digital signal processing. Usage of modern advanced algorithms from this field allowed to design projected device in the first place. This section covers the most importants algorithms that are used in this project.
	
	Digital signal processing is an area of science and engineering that has developed rapidly over the past 40 years as a result of significant advances in digital computer technology. Today, many of the signal processing tasks that were conventionally performed by analog means are now realized by less expensive digital hardware \cite{cite:2}.
	
	To perform the processing digitally, there is need for conversion between analog signal and digital signal. This is done by interface called analog-to-digital (A/D) converter, which yields a digital signal as it's output that is appropriate as an input to the digital processor \cite{cite:2,cite:3}.
	
\subsection{Discrete Fourier Transform}
To perform frequency analysis on a discrete-time signal ${x[n]}$, we convert the time-domain sequence to an equivalent frequency-domain representation. This conversion is obtain by Discrete Fourier Transform that can be algebraically formulated as follows \cite{cite:2,cite:3}.
	
Given N consecutive samples $x[n], 0 \leq n \leq N-1$ of a periodic or aperiodic sequence, the N-point Discrete Fourier Transform(DFT) $X[k], 0 \leq k \leq N-1$ is defined by
\begin{equation} \label{eq:DFT}
X[k]=\sum_{k=0}^{N-1}x[n]e^{-j \frac{2 \pi}{N} kn}.
\end{equation}
Given $N$ DFT coefficients $X[k], 0 \leq k \leq N-1$, we can recover the N sample values of sequence $x[n], 0 \leq n \leq N-1$ using Inverse Discrete Fourier Transform (IDFT) given by
\begin{equation} \label{eq:IDFT}
x[n]=\frac{1}{N} \sum_{k=0}^{N-1}X[k]e^{j \frac{2 \pi}{N} kn}.
\end{equation}
If $x[n]$ has infinite duration, the frequency samples  $X[2 \pi k/ N], k=0, 1, ..., N-1$ correspond to a periodic sequence $x_{p}[n]$ of period N, which is an aliased version of $x[n]$. When the sequence $x[n]$ has finite duration of length $L \leq N$, then  $x_{p}[n]$ is simply a periodic repetition of $x[n]$.

Understanding properties of DFT is critical for application of the transformation to practical problems. List of the main DFT properties contains:

- Linearity,

- Periodicity,

- Complex Conjugate,

- Circular Convolution,

- DFT and the z-transform.

For detailed description of DFT properties see \cite{cite:2,cite:RT_DSP}.

The operation of selecting a finite number of samples called windowing is equivalent to multiplying the actual sequence $x[n]$ defined in range $-\inf < n < \inf$, by a finite-length sequence $w[n]$ called window. Using simplest rectangular windowing (truncation) on signal, can cause effect called \textit{leakage}, which transfers power from frequency bands that contain large amount of signal power into bands that contain only little. This may create "false" peaks, peaks at wrong frequencies or changes the amplitude of existing peaks. Another effect of time-windowing is \textit{smearing}. Which causes spread of spectrum accordingly to the width of the mainlobe of the window spectrum. This result in loss of resolution \cite{cite:3} .
% Image according to ADSP 400/ obr 7.23
Therefore a "good" window should have low level sidelobes and a narrow mainlobe tu minimize both of these effects. There are four most known windows used for time-windowing: 

	- Rectangular,
	
	- Triangular (or Bartlett),
	
	- Hann,
	
	- Hamming.
	
Their differences can be (as shown in image XXX) relays in different width of mainlobe, peak sidelobe level.
% Image according to ADSP 406/ obr 7.26 

%The length N of the DFT should be larger than L = T0/T to obtain good visual representation of DTFT. If we set N to power of two N=2^Q, fft calculation %done microcontroller is executed faster.
\subsection{Fast Fourier Transform}
\subsection{Power spectral density}
Consider a signal $x[n]$ of length $N$ with DFT $X[k]$
\section{Digital Processing Techniques}

\subsection{Spectrograms}

\section{Embedded systems and microcontrollers}
\subsection{Embedded systems}
\subsection{Real-time constraints}
\subsection{ARM M4}

\chapter{Practical Implementation}

\section{Description of Algorithm for Uneveness Analysis}
\subsection{Flowchart Diagram}
\section{Description of Designed System}
\subsection{Block Diagram}
\section{Implementation of Software}
\subsection{Filtration}
%For practical spectral analysis like this, we deal with finite duration discrete-time signals, whose spectrum is given by the DTFT. The N-point FFT is merely used to evaluate samples of the DTFT at N equally spaced frequencies omega=2pik/N, 0 leq n leq N-1. Therefore we needed to use good window (like Hann) and oversample DTFT.
\subsection{Detection of Defects}
\subsection{Automatic Evaulation}
\subsection{Example of Designed Application}
\section{Implementation of Hardware}
\subsection{Electronic Circuits Design}
\subsection{Description of possible sensors}
\subsection{Designed Prototype}
\chapter{Conclusions}


\medskip

\begin{proof}\begin{enumerate} \item[8] Bla \item Blo \end{enumerate} \end{proof}

\appendix

\printindex

\appendix

%\bibliographystyle{amsalpha}
\bibliographystyle{siam}
\bibliography{ctuDIP_biblio}

\ctutemplate{specification.as.chapter}

\end{document}