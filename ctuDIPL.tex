% arara: pdflatex: { synctex: yes }
% arara: makeindex: { style: ctuthesis }
% arara: bibtex

% The class takes all the key=value arguments that \ctusetup does,
% and a couple more: draft and oneside
\documentclass[twoside]{ctuthesis}


\ctusetup{
	preprint = \ctuverlog,
%	mainlanguage = english,
	titlelanguage = english,
	mainlanguage = english,
	otherlanguages = {czech,english},
	title-czech = {Analýza periodických vad textilních vláken ve frekvenční oblasti},
	title-english = {Analysis of the textile fibers unevenness in frequency domain},
	subtitle-czech = {},
	subtitle-english = {},
	doctype = M,
	faculty = F3,
	department-czech = {Katedra měření},
	department-english = {Department of Measurements},
	author = {Ondřej Renza},
	supervisor = {Ing. Jakub Parák},
%	supervisor-address = {Ústav X, \\ Uliční 5, \\ Praha 99},
%	supervisor-specialist = {John Doe},
	fieldofstudy-english = {Sensors and Instrumentation},
	subfieldofstudy-english = {Cybernetics and Robotics},
	fieldofstudy-czech = {Senzory a přístrojová technika},
	subfieldofstudy-czech = {Kybernetika a Robotika},
	keywords-czech = {zpracování digitálního signálu, textilní vady, nerovnoměrnost},
	keywords-english = {digital signal processing, textile defects, uneveness},
	day = 15,
	month = 5,
	year = 2016,
	specification-file = {ctutest-zadani.pdf},
%	front-specification = true,
%	front-list-of-figures = false,
%	front-list-of-tables = false,
%	monochrome = true,
%	layout-short = true,
}

\ctuprocess

\addto\ctucaptionsczech{%
	\def\supervisorname{Vedoucí}%
	\def\subfieldofstudyname{Studijní program}%
}

\ctutemplateset{maketitle twocolumn default}{
	\begin{twocolumnfrontmatterpage}
		\ctutemplate{twocolumn.thanks}
		\ctutemplate{twocolumn.declaration}
		\ctutemplate{twocolumn.abstract.in.titlelanguage}
		\ctutemplate{twocolumn.abstract.in.secondlanguage}
		\ctutemplate{twocolumn.tableofcontents}
		\ctutemplate{twocolumn.listoffigures}
	\end{twocolumnfrontmatterpage}
}

% Theorem declarations, this is the reasonable default, anybody can do what they wish.
% If you prefer theorems in italics rather than slanted, use \theoremstyle{plainit}
\theoremstyle{plain}
\newtheorem{theorem}{Theorem}[chapter]
\newtheorem{corollary}[theorem]{Corollary}
\newtheorem{lemma}[theorem]{Lemma}
\newtheorem{proposition}[theorem]{Proposition}

\theoremstyle{definition}
\newtheorem{definition}[theorem]{Definition}
\newtheorem{example}[theorem]{Example}
\newtheorem{conjecture}[theorem]{Conjecture}

\theoremstyle{note}
\newtheorem*{remark*}{Remark}
\newtheorem{remark}[theorem]{Remark}

% Abstract in Czech
\begin{abstract-czech}
To be done.
\end{abstract-czech}

% Abstract in English
\begin{abstract-english}
 To be done.
\end{abstract-english}

% Acknowledgements / Podekovani
\begin{thanks}
Děkuji.
\end{thanks}

% Declaration / Prohlaseni
\begin{declaration}
Prohlašuji, že jsem předloženou práci vypracoval samostatně, a že jsem uvedl veškerou použitou literaturu.

V Praze, \ctufield{day}.~\monthinlanguage{title}~\ctufield{year}
\end{declaration}

% Only for testing purposes
\listfiles
\usepackage[pagewise]{lineno}
\usepackage{lipsum,blindtext}
\usepackage{mathrsfs} % provides \mathscr used in the ridiculous examples
\usepackage{enumitem}

\begin{document}
	
\maketitle

\chapter{Introduction}
%\chapter{Introduction}
Textile manufacturing has always been important industry field. A major part of this industry is formed by a process called spinning, where twisting strands of fibers together form yarn. The modern spinners - textile machines, that execute the process of spinning - have been significantly improved and now they reach a high level of automation. This allows not only faster and cheaper production but also more focus on quality of the produced textile yarn. 

The quality of the yarn could devaluate the final product by creating defects, such as rapid changes in color or thickness etc., in the textile material. Even with modern technologies, it is still impossible to produce yarn without any defects. We can't prevent yarn defects by carefully selecting and preprocessing fiber because some defects can be created by spinning process itself. Out of many types of defects, this thesis is focused on the analysis of yarn unevenness (also called yarn irregularity). This is describing yarn with a diameter that is not even along its length, but it is changing its value periodically. We can measure this defect in the form of mass variation per unit length.

Designed system is not aiming to improve the quality of spun yarn, but to monitor quality (specifically unevenness) of produced yarn during the spinning process. Due to this monitoring, it is possible to stop spinning process if a defective yarn is detected. This allows the operator to resolve the issues that caused it, e.g. by replacing spinner fiber source with new one and reconnecting different yarn endings.

The project - described in this thesis - has been made in cooperation with company Rieter CZ s.r.o, who provided the device requirements, critical measurement data, and other important information. The goal of the project was to research and develop an algorithm for analysis and detection of yarn unevenness by using spectrograms and to design an embedded system capable of measuring a diameter of yarn while fulfilling the required time constraints and implement detection algorithm. The system is required to be controlled by ARM M4 microcontroller. Thus, the algorithm has to be implemented in a way that takes in consideration memory and computational limitation of such microcontrollers.

A very important specification was the requirement to analyse quality of two textile fiber types: the  yarn and the sliver. Where sliver is the input textile fiber for the spinning process and the yarn is its final product. The both of which has significant physical differences. Mainly they differ in size because yarn diameter is usually in a range of micrometers and sliver diameter is in a range from millimeters to centimeters. The diameter of the sliver is measured on combing machine, which precedes the spinning process. This requires different measuring system and filtration processing therefore, two systems and algorithms were designed for quality analysis of each fiber type. Their core content is equal but there are some major differences, which are described later in this thesis. The most significant difference is that processing of signal representing sliver diameter requires much more advanced techniques of digital signal processing. This is necessary due to a strong presence of periodical artefacts on sliver diameter caused by machine preprocessing. This type of diameter fluctuations has to be distinguished from the actual sliver unevenness, which is task for complicated digital filtration in a frequency domain.

\chapter{Theoretical Introduction}
Before the process of software and hardware development could begin, detailed theoretical research had to be done. Topics of the research include textile manufacturing, combing process, spinning process and textile fiber defects to understand better device requirements. Another step of research was focused on digital signal processing techniques that could be used on the project, mainly spectrogram estimation and it's calculation using Fast Fourier Transform, together with possible filtration algorithms. Another research topic was aimed to cover embedded systems and specifically micro-controller usage and it's real-time constraints.
\section{Textile Engineering}
Goal of textile manufacturing is to make fabric from textile fibers, which can be then used for clothes. This process can be separated in several stages:
\begin{itemize}
	\setlength{\itemsep}{5pt}
\item Preparatory Processes - prepares the textile fiber for spinning process by blending, carding and combing,

\item Spinning - fibers are spun into yarns,

\item Knitting or Weaving - yarns becomes fabric,

\item Finishing - fabric is transformed into clothes etc.
\end{itemize}
In regards to the topic of this thesis only two - the spinning and combing processes - are described.
\subsection{Textile Fibers}
Fibers are the basis for all textiles. We distinguished the two main types: natural fibers and synthetic fibers. The natural fibers are:
\begin{itemize}
	\setlength{\itemsep}{5pt}
\item Cotton - from the cotton plant,
\item Linen - from the flax plant,
\item Wool - from sheep and
\item Silk - from silkworms.
\end{itemize}
Examples of widely used synthetic fibers are:
\begin{itemize}
	\setlength{\itemsep}{5pt}
\item Viscose - from pine trees and petrochemicals,

\item Acrylic, nylon, and polyester - from oil and coal. 
\end{itemize}
The cotton is the most important natural fiber and analysis of the quality in this thesis is aimed specifically at cotton manufacturing. 
The fibers can have two main forms during the manufacturing process - the sliver and the yarn. The sliver is created by carding the fiber. In this process textile fibers are separated and then joined together into a loose strand of 1 cm to 4 cm in diameter. In the end of textile manufacturing process an yarn is created. It is a textile fiber with significantly smaller diameter, in comparison to sliver, usually in a range of micrometers \cite{cite:FoFF}. 
\subsection{Combing Process}
Combing is a preparation process during textile manufacturing. It is sub-part of a cleaning process which precedes the spinning process. Cotton contains a lot of impurities such as dirt, dust, foreign materials, neps and very short fibers. All of these should be eliminated by cleaning process. The combing process removes mainly short fibers and neps in sliver, which helps to produce stronger and cleaner yarn.

Combing is used in a production of medium-fine or fine yarns, where the quality of yarn is important. This quality improvement is at a cost of loss of raw material and high expenses for buying and operating the combing machines. 

Comber (see Image 1) consists of three main parts: 
\begin{itemize}
	\setlength{\itemsep}{5pt}
\item The Feed

\item The Nipper

\item The Comb
\end{itemize}
The process of removing the impurities is formed by attaching the input fibre in form of sliver to the feed roller

An input of this process is textile sliver (described in chapter xx).
\subsection{Spinning Process}
\label{spinningProcesses}
The term “spinning” in this context refers to the process that executes conversion of a large quantity of individual unordered fibers of relatively short length into a linear, ordered product of very great lengths by using spinning machines. There are three main methods of executing process of spinning:
\begin{itemize}
	\setlength{\itemsep}{5pt}
\item Ring Spinning,

\item Rotor Spinning and

\item Air-jet Spinning.
\end{itemize}
All of these systems yields yarn with different structures and properties. Each system has its advantages and limitations in terms of technical feasibility and economic viability.
\section{Overview of Yarn Quality Sensors}
Importance of yarn quality on final product became clear in 1950s when first electronic yarn quality sensors were invented. Since then many principles were used in detection of yarn defects - optical, mechanical or even chemical \cite{cite:1}.

Sensors of yarn quality are in textile industry often called \textit{yarn clearer}. This term was created for first such devices because goal of yarn clearers wasn't only to discover possible yarn defects but also to immediately remove them. Today, yarn clearers analyse defects that are much more complex (given their properties such as periodicity etc.) and difficult to be immediately removed. Such defects are invisible to naked eye on single yarn and they appear only when turned to fabric.

Following overview is concerned with yarn quality sensors that have similar purpose and functions as the developed device. Therefore, only online yarn clearers that check quality of spun yarn on every spinning point in real-time are listed. All of the following devices are also aimed to be used on the yarn spun by a rotor spinners (see \ref{spinningProcesses}).

\subsubsection{Uster Quantum 3}
Uster Quantum 3 is modern, state of art yarn clearer, which provide detection of many types of defects and includes several interesting features:

\begin{itemize}
	\setlength{\itemsep}{5pt}
	\item full yarn body display,
	\item foreign matter sensor with multicolored light sources,
	\item polypropylene detection,
	\item detection of moiré,
	\item unevenness calculation CV\%,
	\item IPI classification,
	\item calculation of hairiness,
	\item spectrograms calculation
\end{itemize}

Uster Quantum 3 has been developed by company Uster Technologies, which is developing yarn clearers for 30 years. Their devices often feature application of new technologies and they offer 
\subsection{State of Art}

\section{Digital Signal Processing}
The largest part of work on this project is oriented on digital signal processing (DSP). Usage of modern advanced algorithms from this field allowed designing projected device in the first place. This section covers the most important algorithms that are used in this project.
	
	Digital signal processing is an area of science and engineering that has developed rapidly over the past 40 years as a result of significant advances in digital computer technology. Today, many of the signal processing tasks that were conventionally performed by analog means are now realized by less expensive digital hardware \cite{cite:2}.
	
	To perform the processing digitally, there is need for the conversion between an analog signal and digital signal. This is done by an interface called analog-to-digital (A/D) converter, which yields a digital signal as it's output that is appropriate as an input to the digital processor \cite{cite:2,cite:3}.
	
\subsection{Discrete Fourier Transform}
To perform frequency analysis on a discrete-time signal ${x[n]}$, we convert the time-domain sequence to an equivalent frequency-domain representation. This conversion is obtained by Discrete Fourier Transform that can be algebraically formulated as (according to \cite{cite:2,cite:3}).
	
Given N consecutive samples $x[n], 0 \leq n \leq N-1$ of a periodic or aperiodic sequence, the N-point Discrete Fourier Transform(DFT) $X[k], 0 \leq k \leq N-1$ is defined by
\begin{equation} \label{eq:DFT}
X[k]=\sum_{k=0}^{N-1}x[n]e^{-j \frac{2 \pi}{N} kn}.
\end{equation}
Given $N$ DFT coefficients $X[k], 0 \leq k \leq N-1$, we can recover the N sample values of sequence $x[n], 0 \leq n \leq N-1$ using Inverse Discrete Fourier Transform (IDFT) given by
\begin{equation} \label{eq:IDFT}
x[n]=\frac{1}{N} \sum_{k=0}^{N-1}X[k]e^{j \frac{2 \pi}{N} kn}.
\end{equation}
If $x[n]$ has infinite duration, the frequency samples  $X[2 \pi k/ N], k=0, 1, ..., N-1$ correspond to a periodic sequence $x_{p}[n]$ of period N, which is an aliased version of $x[n]$. When the sequence $x[n]$ has finite duration of length $L \leq N$, then  $x_{p}[n]$ is simply a periodic repetition of $x[n]$.

The DFT defined in (\ref{eq:DFT}) can also be rewritten as
\begin{equation} \label{eq:DFT2}
X[k]=\sum_{k=0}^{N-1}x[n]W^{kn}_{N},\; k = 0, 1, ..., N-1,
\end{equation}
where
\begin{equation} \label{eq:Twiddle}
W^{kn}_{N}=e^{-j \frac{2 \pi}{N} kn}=\cos(\frac{2\pi kn}{N})-j\sin(\frac{2\pi kn}{N}), \;0\leq k,n\leq N-1
\end{equation}
The parameters $W^{kn}_{N}$ are called the twiddle factors \cite{cite:RT_DSP}.
\par
Understanding properties of DFT is critical for application of the transformation to practical problems. List of the main DFT properties contains:
\begin{itemize}
	\setlength{\itemsep}{5pt}
\item Linearity,

\item Periodicity,

\item Complex Conjugate,

\item Circular Convolution,

\item DFT and the z-transform.
\end{itemize}
For detailed description of DFT properties see \cite{cite:2,cite:RT_DSP}.

The operation of selecting a finite number of samples called windowing is equivalent to multiplying the actual sequence $x[n]$ defined in a range $-\infty < n < \infty$, by a finite-length sequence $w[n]$ called window. Using simplest rectangular windowing (truncation) on a signal, can cause an effect called \textit{leakage}, which transfers power from frequency bands that contain a large amount of signal power into bands that contain only a little. This may create "false" peaks, peaks at wrong frequencies or changes the amplitude of existing peaks. 

Another effect of time-windowing is \textit{smearing}. Which causes a spread of spectrum accordingly to the width of the mainlobe of the window spectrum. This result in loss of resolution \cite{cite:3} .
% Image according to ADSP 400/ obr 7.23

Therefore, a "good" window should have low-level sidelobes and a narrow mainlobe to minimize both of these effects. There are four most known windows used for time-windowing: 
\begin{itemize}
	 \setlength{\itemsep}{5pt}
\item Rectangular,
	
\item Triangular (or Bartlett),
	
\item Hann,
	
\item Hamming.
\end{itemize}	
Their differences (as shown in image XXX) relays in a different width of mainlobe and peak sidelobe level.
% Image according to ADSP 406/ obr 7.26 

%The length N of the DFT should be larger than L = T0/T to obtain good visual representation of DTFT. If we set N to power of two N=2^Q, fft calculation %done microcontroller is executed faster.
\subsection{Fast Fourier Transform}
\label{sec:FFT}
Difficulty in using the DFT for practical applications is its high computational requirements. Direct computation of the N-point DFT requrires computational cost of $N^2$. However class of efficient DFT algorithms called \textit{Fast Fourier Transform (FFT)} has computational cost proportional to $Nlog_{2}N$ \cite{cite:RT_DSP,cite:3}.
\par 
Decimation-in-time FFT algorithms are based in splitting the N-point DFT summation into two summations, that one sum over the even-indexed points of $x[n]$ and another sum over the odd-indexed points of $x[n]$.
Therefore, we obtain
\begin{equation} \label{eq:decimInTime1}
\begin{aligned}
X[k] &= \sum_{n=0}^{N-1}x[n]W^{kn}_{N}, \; k=0,1,...,N-1\\
     &= \sum_{m=0}^{N/2-1}x[2m]W^{k(2m)}_{N} + W^{k}_{N}\sum_{m=0}^{N/2-1}x[2m+1]W^{k(2m)}_{N}
     \end{aligned}
\end{equation}
Dividing sequence $x[n]$ we get two shorter sequences:
\begin{equation} \label{eq:decimatedSequencesA}
a[n]=x[2n],\qquad n=0, 1, ..., N/2 -1
\end{equation}
\begin{equation} \label{eq:decimatedSequencesB}
b[n]=x[2n+1],\qquad n=0, 1, ..., N/2 -1
\end{equation}
Shorter sequences are obtained by \textit{decimating}
\footnote{Decimation of a signal with sampling rate $f_{s}$ by a integer factor $D$ results in the lower sampling rate $f'_{s}=f_{s}/D$.}
 the sequence $x[n]$, thus, this FFT algorithm is called decimation-in-time.
Substituting definitions \ref{eq:decimatedSequencesA} and \ref{eq:decimatedSequencesB} into \ref{eq:decimInTime1} yields
\begin{equation} \label{eq:fft2_A}
A[k]=\sum_{m=0}^{N/2-1}a[m]W^{km}_{N/2},\qquad k=0, 1, ..., N/2 -1
\end{equation}
\begin{equation} \label{eq:fft2_B}
B[k]=\sum_{m=0}^{N/2-1}a[m]W^{km}_{N/2},\qquad k=0, 1, ..., N/2 -1
\end{equation}
where $A[k]$ and $B[k]$ are $N/2$-point DFTs \cite{cite:3,cite:2}.

Thus, we can calculate $N$-point DFT $X[k]$ from the $N/2$-point DFTs $A[k]$ and $B[k]$ (\ref{eq:fft2_A}, \ref{eq:fft2_B}) using the following merging formulas
\begin{equation} \label{eq:fft3_A}
X[k]=A[k] + W^{k}_{N}B[k],\qquad k=0, 1, ..., N/2 -1
\end{equation}
\begin{equation} \label{eq:fft3_B}
X[k+\frac{N}{2}]=A[k] - W^{k}_{N}B[k],\qquad k=0, 1, ..., N/2 -1
\end{equation}
These formulas (\ref{eq:fft3_A}, \ref{eq:fft3_B}) can be applied to any FFT of even length \cite{cite:3}.

This procedure is shown in Figure XXX (\cite{cite:3}). The displayed structure in the figure is called the butterfly network. Each butterfly consist of just a single complex multiplication by the twiddle factor $W^{k}_{N}$, one addition and one substraction. 

An example for $N=8$ is show in Figure XXX. Each $N/2$-point DFT can be computed by two smaller $N/4$-point DFTs. By repeating the same process, we will obtain a set of two-point DFTs, which is illustrated in Figure XXX \cite{cite:RT_DSP}.

FFT algorithm \textit{decimation-in-frequency} is similar to the decimation-in-time, with important differences, that the decomposition and symmetry relationships are reversed. The bit reversal occurs at the output instead of the input and the order of the output samples $X[k]$ will be rearranged \cite{cite:RT_DSP}.

The FFT algorithms shown in the previous paragraphs can be modified to calculate the inverse FFT (IFFT).

\subsection{Power spectral density}
%Consider a signal $x[n]$ of length $N$ with DFT $X[k]$ 
Energy Spectral Density $S(\omega)$ defined as \ref{eq:EnergySD}
\begin{equation} \label{eq:EnergySD}
S(\omega)=\left|Y(\omega)\right|^2
\end{equation}
Which can be obtained from Parseval's theorem (\ref{eq:Parseval})
\begin{equation} \label{eq:Parseval}
\sum_{t=-\infty}^{\infty}\left|y(t)\right|^2 = \frac{1}{2\pi}\int_{-\pi}^{\pi}S(\omega)d\omega
\end{equation}
This equality shows that $S(\omega)$ represents the distribution of sequence energy as a function of frequency. For this reason, $S(\omega)$ is called the \textit{energy spectral density}. 
Most of the signals in practical applications are such that their variation in the future cannot be known. It is only possible to make probabilistic statement about the variation. Such sequences are called \textit{random signals}. 

A random signal usually has finite average power, therefore, we can use average power spectral density for its characterization. Which is, for simplicity, better known under name \textit{power spectral density} (PSD). As shown in \cite{cite:SAoS} is defined as
\begin{equation} \label{eq:PSD1}
\phi(\omega) = \sum_{t=-\infty}^{\infty}r(k)e^{-i\omega k}
\end{equation}
where $r(k)$ is auto covariance function $r(k) = E[y_{t}y_{t+k}]$.
The PSD is very useful in the analysis of random signals since it provides us with information about distribution of the average power over the frequency. There are several different methods for estimating the PSD. 

Methods that require direct use of finite signal for purposes of autocorrelation calculation are called \textit{non-parametric} methods. On the other hand, methods that rely on a model for signal generation are call \textit{parametric} methods \cite{cite:RT_DSP}. Widely used non-parametric methods are

\begin{itemize}
	\setlength{\itemsep}{5pt}
\item Bartlett method,
\item Blackman-Turkey method,
\item Welch method or
\item Danielle method.
\end{itemize}

They differ mainly in the resolution and variance level of the result. For detailed description of methods see \cite{cite:2}.

\section{Digital Processing Techniques}

\subsection{Spectrograms}

\section{Embedded systems and microcontrollers}
There are many definitions of embedded systems. The one picked for this thesis subjectively seems well accurate in the description of the given term. It goes as follows.

\textit{"An embedded system is a specialized computer system that is usually integrated as part of a larger system. An embedded system consist of a combination of hardware and software components to form a computational engine that will perform a specific function. Unlike desktop systems which are designed to perform a general function, embedded systems are constrained in their application \cite{cite:SE_for_ES}."}

Embedded systems very often perform in reactive and real-time environments, which has to fulfill the real-time constrains that are described in \ref{R-T constrains}. Not satisfying these constrains can cause significant system consequences. If the consequences consist of a degradation of performance, but not failure, the system is referred to as a soft real-time system (e.g. highway car counter). On the contrary, if the consequences are system failure, the system is referred to as a hard real-time system (e.g. a braking system in vehicle). 

Typical embedded system receives information about the surrounding enviroment via sensors and responds with actuators. General block diagram of such system is in Figure XXX.																	
\subsection{Embedded systems}

\subsection{DSP Hardware Options}
DSP algorithms can be implemented on different types of digital hardware. The following are the widely used options for DSP systems:
\begin{itemize}
	\setlength{\itemsep}{5pt}
\item Special-purpose chips such as application-specific integrated circuits (ASICs).
\item Field-programmable gate arrays (FPGAs).
\item General-purpose micro-processors or micro-controllers ($\mu$P/$\mu$C).
\item General-purpose digital signal processors.
\item DSP processors with application-specific hardware accelerators \cite{cite:RT_DSP}.
\end{itemize}

Each hardware (HW) platform has different advantages and constrains for different applications, thus, there is no \textit{best} HW platform that could be used for every practical project. Instead each option should be carefully considered from point of flexibility, required design time, power consumption, performance and cost. 

Characteristic of mentioned hardware options are summarized in Table XXX (\cite{cite:RT_DSP}).

\subsection{Real-time constraints}
\label{R-T constrains}
Generally, a real-time system is one that must process information and produce a response within a specified time, else risk severe consequences, including failure. A real-time DSP system demands that the signal processing time $t_{p}$, must be less than sampling period $T$, that is
\begin{equation} \label{eq:hardRealTime1}
t_{p}+t_{o}<T,
\end{equation}
where $t_{o}$ is overhead time of input-output (I/O) processing.
Thus, this limitation gives constraint to the highest frequency signal that can be processed by DSP systems in sample-by-sample processing, given as

\begin{equation} \label{eq:hardRealTime2}
f_{M} \leq \frac{fs}{2} < \frac{1}{2(t_{p}+t_{o})} \cite{cite:RT_DSP}.
\end{equation}

Using different techniques of processing can reduce the I/O overhead time and increase the performance of the DSP hardware platforms. For example by applying a block-by-block processing, where the I/O operations are handled by DMA controllers, which place data samples in memory buffers \cite{cite:RT_DSP}.

Today, with performance improvement of hardware platforms, it is even possible to calculate FFT (see \ref{sec:FFT}) of 64-points in matter of tenths of milliseconds with low-cost ARM microprocessor (e.g. specifically in $0.16ms$ for microcontrollers of STM32F1xx series according to \cite{cite:STM32_DSP_library}).
\subsection{ARM Cortex-M3}
First manufactured microprocessor of Cortex series was ARM Cortex-M3. It was developed in 2006 and its target group of application were 32-bit microcontrollers. Its main advantage is an excellent efficiency which yields high performance and a low energy consumption without need for a very high system clock frequency. It is based on architecture \textit{ARMv7}.

An ARMv7 architecture was developed as modern type of a general architecture that could be used for low-level microprocessors as well as for high-performance application processors. Design of the architecture is split in three main profiles:
\begin{itemize}
	\setlength{\itemsep}{5pt}
	\item Profile A - aimed on application processors for performance-intensive systems capable of using embedded operational systems (such as Symbian, Android or Linux Embedded).
	\item Profile R - aimed on high-performance processors for real-time applications.
	\item Profile M - aimed on wide-range processors for deeply embedded applications \cite{cite:ARM-M3}.
\end{itemize}

\subsubsection{Instruction Set}
There are two instruction sets supported by ARM cores in general: the ARM instructions which are 32-bit and the Thumb instructions which are 16-bit. During the execution, the processor can switch either to the ARM state or to the Thumb state accordingly to the instruction set it is currently using.

The ARMv7 architecture is using new instruction set called a Thumb-2 which consist of the 32-bit Thumb instructions and also 16-bit Thumb instructions. This yields big improvement from the perspective of the ease of use, performance and code size. It allows execution of complex operation in state Thumb which increases the effectivity \cite{cite:ARM-M3}.

\chapter{Practical Implementation}

\section{Description of Algorithm for Uneveness Analysis}
\subsection{Flowchart Diagram}
\section{Description of Designed System}
\subsection{Block Diagram}
\section{Implementation of Software}
\subsection{Filtration}
%For practical spectral analysis like this, we deal with finite duration discrete-time signals, whose spectrum is given by the DTFT. The N-point FFT is merely used to evaluate samples of the DTFT at N equally spaced frequencies omega=2pik/N, 0 leq n leq N-1. Therefore we needed to use good window (like Hann) and oversample DTFT.
\subsection{Detection of Defects}
\subsection{Automatic Evaulation}
\subsection{Example of Designed Application}
\section{Implementation of Hardware}
\subsection{Electronic Circuits Design}
\subsection{Description of possible sensors}
\subsection{Designed Prototype}
\chapter{Conclusions}


\medskip

\begin{proof}\begin{enumerate} \item[8] Bla \item Blo \end{enumerate} \end{proof}

\appendix

\printindex

\appendix

%\bibliographystyle{amsalpha}
\bibliographystyle{siam}
\bibliography{ctuDIP_biblio}

\ctutemplate{specification.as.chapter}

\end{document}